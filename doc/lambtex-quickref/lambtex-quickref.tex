\documentclass[10pt]{article}

\usepackage{geometry}                           % To define page size, margins, etc.
\usepackage[pdftex]{hyperref}                   % For hyper-references in the generated PDF.
\usepackage[hyperref]{xcolor}                   % To use colours in the document.
\usepackage[pdftex]{graphicx}                   % To import graphics.
\usepackage{epstopdf}                           % For on-the-fly conversion of eps to pdf.
\usepackage[british]{babel}                     % For British English hyphenation patterns.
\usepackage[latin1]{inputenc}                   % For Latin1 (ISO-8859-1) input encoding.
\usepackage[T1]{fontenc}                        % For T1 character encoding.
\usepackage{textcomp}                           % For T1 character encoding.
\usepackage{ae}                                 % For T1 encoding using Type 1 fonts.
\usepackage{mathptmx}                           % For Type 1 "Times Roman" serif text and math.
\usepackage[scaled=0.92]{helvet}                % For Type 1 "Helvetica" sans serif text.
\usepackage{courier}                            % For Type 1 "Courier" monospaced text.
\usepackage{xspace}                             % For automatic handling of spaces after macros.
\usepackage{booktabs}				% For better tables.
\usepackage{multirow}				% For merging rows.
\usepackage{soul}				% For strike-through.
\usepackage[titles]{tocloft}                    % For Table of Contents customisation.
\usepackage{subfig}                             % For subfigures.
\usepackage{fancyvrb}                           % For fancy verbatim environments.
\usepackage[stable]{footmisc}                   % For stable footnotes.
\usepackage{float}                              % For extra control over floats.
\usepackage{titlesec}                           % To configure sections headings.
\usepackage[nottoc]{tocbibind}                  % To put the Bibliography in the TOC.
\usepackage{units}                              % To typeset units.

\geometry{dvipdfm,a4paper,landscape,centering,tmargin=2.5cm,bmargin=2.5cm,lmargin=2.5cm,rmargin=2.5cm,nomarginpar}

\definecolor{codefront}{rgb}{0,0,0}
\definecolor{codeback}{rgb}{1,1,0.8}
\definecolor{codeframe}{rgb}{0,0,0}
\definecolor{Brick}{HTML}{7B0C00}
\definecolor{RoyalBlue}{cmyk}{1,0.50,0,0}

\definecolor{no}{rgb}{0.9,0.8,0.8}
\definecolor{opt}{rgb}{0,0.5,0}
\definecolor{dep}{rgb}{0.8,0.4,0}

\definecolor{composition}{rgb}{0,0.3,0}
\definecolor{manuscript}{rgb}{0,0,0.3}

\hypersetup{%
        linkcolor=RoyalBlue,
        urlcolor=RoyalBlue,
        plainpages=false,
        pdfpagelabels,
        hyperindex,
        colorlinks=true}

\hypersetup{%
        pdftitle={Asymptote and LaTeX: An Integration Guide},
        pdfauthor={Dario Teixeira (dario.teixeira@yahoo.com)},
        pdfsubject={asymptote},
        pdfkeywords={typesetting, mathematical illustrations, asymptote, latex}}

\begin{document}

\newcommand{\no}[0]{\textcolor{no}{---}}
\newcommand{\opt}[0]{\textcolor{opt}{opt}}
\newcommand{\dep}[0]{\textcolor{dep}{dep\textsuperscript{1}}}
\newcommand{\depz}[0]{\textcolor{dep}{dep\textsuperscript{2}}}

\newcommand{\C}[0]{\textcolor{composition}{C}}
\newcommand{\M}[0]{\textcolor{manuscript}{M}}

\newcommand{\op}[1]{\textbf{#1}}
\newcommand{\hd}[1]{\textbf{#1}}
\newcommand{\simc}[1]{\textcolor{composition}{\texttt{\bfseries$\backslash$#1}}}
\newcommand{\simm}[1]{\textcolor{manuscript}{\texttt{\bfseries$\backslash$#1}}}
\newcommand{\envc}[1]{\textcolor{composition}{\texttt{\bfseries#1}}}
\newcommand{\envm}[1]{\textcolor{manuscript}{\texttt{\bfseries#1}}}
\newcommand{\textsubscript}[1]{\ensuremath{_{\textrm{#1}}}}

\begin{tabular}{rlcllllp{30em}}

\toprule

			&				&	& \multicolumn{4}{c}{\hd{Parameters}} & \\

\cmidrule{4-7}\\

\hd{Command}		& \hd{Synonyms}			& \hd{T}& \hd{Mandatory}		& \hd{Order}	& \hd{Label}	& \hd{Extra}
& \hd{Description}\\

\midrule

\simc{br}		& \no				& \C	& \no				& \no		& \no		& \no
& Inserts a line break within the same paragraph.\\

\simc{bold}		& \simc{b}, \simc{strong}	& \C	& \{inline\}			& \no		& \no		& \no
& Sets the inline parameter in \textbf{bold} font.\\

\simc{emph}		& \simc{i}, \simc{em}		& \C	& \{inline\}			& \no		& \no		& \no
& Sets the inline parameter in \emph{emphasised} font.\\

\simc{mono}		& \simc{m}, \simc{tt}		& \C	& \{inline\}			& \no		& \no		& \no
& Sets the inline parameter in \texttt{monospaced} (teletype) font.\\

\simc{caps}		& \no				& \C	& \{inline\}			& \no		& \no		& \no
& Sets the inline parameter in \textsc{small caps}.\\

\simc{thru}		& \no				& \C	& \{inline\}			& \no		& \no		& \no
& Sets the inline parameter in \st{strike-throuh} style.\\

\simc{sup}		& \no				& \C	& \{inline\}			& \no		& \no		& \no
& Sets the inline parameter as \textsuperscript{superscript} text.\\

\simc{sub}		& \no				& \C	& \{inline\}			& \no		& \no		& \no
& Sets the inline parameter as \textsubscript{subscript} text.\\

\simc{mbox}		& \no				& \C	& \{inline\}			& \no		& \no		& \no
& Specifies that the inline parameter should not be split across lines.\\

\simc{link}		& \simc{a}			& \C	& \{uri\}\{inline\op{?}\}	& \no		& \no		& \no
& Creates a link to the specified \textsc{URI} using the inline text for display.\\

\simm{cite}		& \no				& \M	& \{label\}			& \no		& \no		& \no
& Creates a link to the specified entry in the bibliography.\\

\simm{see}		& \no				& \M	& \{label\}			& \no		& \no		& \no
& Creates a link to the specified document note.\\

\simm{ref}		& \no				& \M	& \{label\}			& \no		& \no		& \no
& Creates a manual reference to the specified element.\\

\simm{sref}		& \no				& \M	& \{label\}			& \no		& \no		& \no
& Creates an automatic reference to the specified element.\\

\simm{mref}		& \no				& \M	& \{label\}\{inline\}		& \no		& \no		& \no
& Creates a reference to the specified element using the custom inline text for display.\\

\bottomrule

\end{tabular}

\begin{tabular}{rlcllllp{30em}}

\toprule

			&				&	& \multicolumn{4}{c}{\hd{Parameters}} & \\

\cmidrule{4-7}\\

\hd{Command}		& \hd{Synonyms}			& \hd{T}& \hd{Mandatory}		& \hd{Order}	& \hd{Label}	& \hd{Extra}
& \hd{Description}\\

\midrule

\simc{item}		& \simc{li}			& \C	& \no				& \no		& \no		& \no
& Declares a new item in an ordered or unordered list.\\

\simc{describe}		& \simc{dt}			& \C	& \{inline\}			& \no		& \no		& \no
& Declares the header for a new item in a description list.\\

\simc{bitmap}		& \no				& \C	& \{label\}\{raw\}		& \no		& \no		& align, frame, width
& Inserts a bitmap image.\\

\simc{head}		& \no				& \C	& \no				& \no		& \no		& \no
& Declares the start of the head rows in a tabular environment.\\

\simc{body}		& \no				& \C	& \no				& \no		& \no		& \no
& Declares the start of a block of body rows in a tabular environment.\\

\simc{foot}		& \no				& \C	& \no				& \no		& \no		& \no
& Declares the start of the foot rows in a tabular environment.\\

\simm{caption}		& \no				& \M	& \{inline\}			& \no		& \no		& \no
& Declares the caption for a wrapper (equation, printout, table, or figure).\\

\simm{who}		& \no				& \M	& \{inline\}			& \no		& \no		& \no
& Declares the author in a bibliography entry.\\

\simm{what}		& \no				& \M	& \{inline\}			& \no		& \no		& \no
& Declares the title in a bibliography entry.\\

\simm{where}		& \no				& \M	& \{inline\}			& \no		& \no		& \no
& Declares the resource where the bibliography entry may be located.\\

\simm{part}		& \no				& \M	& \{inline\}			& \depz		& \opt		& \no
& Declares a document part (the highest level division).\\

\simm{appendix}		& \no				& \M	& \no				& \no		& \opt		& \no
& Declares the beginning of the document's appendix.\\

\simm{section}		& \simm{h1}			& \M	& \{inline\}			& \depz		& \opt		& \no
& Declares a new document section.\\

\simm{subsection}	& \simm{h2}			& \M	& \{inline\}			& \depz		& \opt		& \no
& Declares a new document sub-section.\\

\simm{subsubsection}	& \simm{h3}			& \M	& \{inline\}			& \depz		& \opt		& \no
& Declares a new document sub-sub-section.\\

\simm{bibliography}	& \no				& \M	& \no				& \no		& \opt		& \no
& Inserts all bibliography entries.\\

\simm{notes}		&\no				& \M	& \no				& \no		& \opt		& \no
& Inserts all document notes.\\

\simm{toc}		& \no				& \M	& \no				& \no		& \opt		& \no
& Inserts the document's table of contents.\\

\simm{title}		& \no				& \M	& \{inline\}			& \no		& \no		& \no
& Declares the document title.\\

\simm{subtitle}		& \no				& \M	& \{inline\}			& \no		& \no		& \no
& Declares a document subtitle.\\

\simm{rule}		& \simm{hr}			& \M	& \no				& \no		& \no		& \no
& Inserts an horizontal separator.\\

\bottomrule

\end{tabular}

\begin{tabular}{rlcllllp{35em}}

\toprule

			&				&	& \multicolumn{4}{c}{\hd{Parameters}} & \\

\cmidrule{4-7}\\

\hd{Command}		& \hd{Synonyms}			& \hd{T}& \hd{Mandatory}		& \hd{Order}	& \hd{Label}	& \hd{Extra}
& \hd{Description}\\

\midrule

\envc{itemize}		& \envc{ul}			& \C	& \no				& \no		& \no		& bul
& Declares an unordered list.\\

\envc{enumerate}	& \envc{ol}			& \C	& \no				& \no		& \no		& num
& Declares an ordered list.\\

\envc{description}	& \envc{dl}			& \C	& \no				& \no		& \no		& \no
& Declares a description list.\\

\envc{verse}		& \no				& \C	& \no				& \no		& \no		& \no
& Declares a verse block.\\

\envc{quote}		& \no				& \C	& \no				& \no		& \no		& \no
& Declares a quotation block.\\

\envc{mathtex}		& \no				& \C	& \no				& \no		& \no		& align
& Declares a math block in \TeX format.\\

\envc{mathml}		& \no				& \C	& \no				& \no		& \no		& align
& Declares a math block in \textsc{MathML} format.\\

\envc{code}		& \no				& \C	& \no				& \no		& \no		& align
& Declares a verbatim block containing source-code.\\

\envc{tabular}		& \no				& \C	& \{raw\}			& \no		& \no		& align
& Declares a tabular environment.\\

\envc{verbatim}		& \no				& \C	& \no				& \no		& \no		& align
& Declares a verbatim block with \textsc{ascii}-art.\\

\envc{subpage}		& \no				& \C	& \no				& \no		& \no		& align
& Declares a sub-page containing a child document.\\

\envm{pull}		& \no				& \M	& \no				& \no		& \no		& align
& Declares a pull-quote.\\

\envm{boxout}		& \no				& \M	& \{inline\op{?}\}		& \no		& \no		& align
& Declares a boxout with some inline text as title.\\

\envm{equation}		& \no				& \M	& \no				& \opt		& \dep		& \no
& Declares a numbered wrapper around a math block.\\

\envm{printout}		& \no				& \M	& \no				& \opt		& \dep		& \no
& Declares a numbered wrapper around a code block.\\

\envm{table}		& \no				& \M	& \no				& \opt		& \dep		& \no
& Declares a numbered wrapper around a tabular environment.\\

\envm{figure}		& \no				& \M	& \no				& \opt		& \dep		& \no
& Declares a numbered wrapper around a verbatim, bitmap, or sub-page figure.\\

\envm{bib}		& \no				& \M	& \no				& \no		& \no		& \no
& Declares a bibliography entry.\\

\envm{note}		& \no				& \M	& \no				& \no		& \no		& \no
& Declares a document note.\\

\envm{abstract}		& \no				& \M	& \no				& \no		& \no		& \no
& Declares a block of text to be the document's abstract.\\

\bottomrule

\end{tabular}

\end{document}

